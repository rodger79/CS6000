\documentclass[conference]{IEEEtran}
%\renewcommand{\thesubsection}{\thesection.\alph{subsection}}

%\addtolength{\oddsidemargin}{-.875in}
%\addtolength{\evensidemargin}{-.875in}
%\addtolength{\textwidth}{1.75in}
%\addtolength{\topmargin}{-.875in}
%\addtolength{\textheight}{1.75in}
	
\usepackage{bm}
\usepackage{amsmath}
\usepackage{amssymb}
\usepackage{tikz}
\usetikzlibrary{automata,positioning}
\usepackage{url}
\usepackage{float}
\usepackage{setspace}
\usepackage{filecontents,lipsum}
\usepackage[noadjust]{cite}

\begin{document}
%\raggedright
%\doublespacing

\title{Week 4 Journal}
\author{Rodger Byrd}
\maketitle


\section{Topic Map}
For my area I'm looking at Anti-Patterns in code, these are also known as code smells. I've also seen them referred to as Atoms of Confusion and nano patterrns. My topic map is included below in figure \ref{fig:TM}. My thoughts on gaps are some method to demonstrate the found anti-patterns are valid. Also, there seems to be a gap in how to fix anti-patterns and code smells. I've also included my notes on creating the Topic map in Figure 

\subsection{Who is the Main Character}
\subsection{What Character Traits Make them Interesting}
\subsection{What do the Character Need to do or Get (Goal)}
\subsection{Why is That Goal Important (motive)}
\subsection{What Conflicts/Problems Block the Character}
\subsection{Ho do they Create Risk and Danger}
\subsection{What Does the Character Do (Struggles) to Reach Goal}
\subsection{What Sensory Details Will Make the Story Seem Real}

The files for this latex document are in the github repository located at \path{https://github.com/rodger79/CS6000}

Papers that were scanned and trashed are referenced in the bibliography below. 
\nocite{*}
\clearpage


\bibliographystyle{IEEEtran}
\bibliography{references}


\end{document}