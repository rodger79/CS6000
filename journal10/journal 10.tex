\documentclass[conference]{IEEEtran}
%\renewcommand{\thesubsection}{\thesection.\alph{subsection}}

%\addtolength{\oddsidemargin}{-.875in}
%\addtolength{\evensidemargin}{-.875in}
%\addtolength{\textwidth}{1.75in}
%\addtolength{\topmargin}{-.875in}
%\addtolength{\textheight}{1.75in}
	
\usepackage{bm}
\usepackage{amsmath}
\usepackage{amssymb}
\usepackage{tikz}
\usetikzlibrary{automata,positioning}
\usepackage{url}
\usepackage{float}
\usepackage{setspace}
\usepackage{filecontents,lipsum}
\usepackage[noadjust]{cite}
\usepackage{listings}

\begin{document}
%\raggedright
%\doublespacing

\title{Week 10 Journal}
\author{Rodger Byrd}
\maketitle



\section{Journal Entry}
This week finalized and submitted the survey paper on code smells and submitted my AWS video.

\subsection{Most Significant Learning Experiences}
Generally this class provides a very good and broad overview of academic research and I found it to be very thorough and helpful.

One of the things I found very interesting was in reading papers creatively not just critically. 
It changed my mindset while reading papers. 

I found the statistics and experiment desing to be very helpful. 
My advisor asked me previously to perform some T-tests as part of my graduate thesis, and I had no idea what they were. 
Being able to show that the results from your expirements are meaningful is a critical item in research. 
I found your overview of these sections very helpful and it was one of the things I had hoped to get out of the course.


\subsection{Least Significant Learning Experiences}
I think most people who are grad students in computer science should know how to use git so I found that to be mostly review.

The files for this latex document are in the github repository located at \path{https://github.com/rodger79/CS6000}

%Relevan papers are referenced in the bibliography below. 
\nocite{*}
\clearpage


%\bibliographystyle{IEEEtran}
%\bibliography{references}


\end{document}