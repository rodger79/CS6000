\documentclass[conference]{IEEEtran}
%\renewcommand{\thesubsection}{\thesection.\alph{subsection}}

%\addtolength{\oddsidemargin}{-.875in}
%\addtolength{\evensidemargin}{-.875in}
%\addtolength{\textwidth}{1.75in}
%\addtolength{\topmargin}{-.875in}
%\addtolength{\textheight}{1.75in}
	
\usepackage{bm}
\usepackage{amsmath}
\usepackage{amssymb}
\usepackage{tikz}
\usetikzlibrary{automata,positioning}
\usepackage{url}
\usepackage{float}
\usepackage{setspace}
%\usepackage{IEEEtran}
\usepackage{filecontents,lipsum}
\usepackage[noadjust]{cite}

\begin{document}
%\raggedright
%\doublespacing

\title{Week 2 Journal}
\author{Rodger Byrd}
\maketitle


\section{Process}

For this Journal, I used Zotero to track all of my bibliography entries and notes. It was significantly faster and easier than what I did in Journal 1. 
For the "top" journal I chose IET Information Security \cite{noauthor_iet_nodate}. It had 5 issues in 2019 with 10-12 articles per issue. My filed of study is related to both Compute Science and Computer Security so I thought this would include relevant papers for me to read. My process included 3 phases. 
First, quickly browse each paper using my phone as a stopwatch to determine if I wanted to scan or trash the paper with a note about why.
Secondly, I did a 5-10 minute scan of the papers chosen in phase 1 to determine if I wanted to critically and creatively read them.
Lastly, I chose the best 2 papers and read them tracking my notes. The notes are included in the next section.

\section{Raw Notes}
The second section is your raw notes from critical/creative reads as a latex document, and include all the papers in the references. Not expecting/wanting full sentences, grammar, want your raw notes.  If you use handwritten notes, you can take pic and include them as figures. 



\bibliographystyle{IEEEtran}
\bibliography{references}


\end{document}