\documentclass[conference]{IEEEtran}
%\renewcommand{\thesubsection}{\thesection.\alph{subsection}}

%\addtolength{\oddsidemargin}{-.875in}
%\addtolength{\evensidemargin}{-.875in}
%\addtolength{\textwidth}{1.75in}
%\addtolength{\topmargin}{-.875in}
%\addtolength{\textheight}{1.75in}
	
\usepackage{bm}
\usepackage{amsmath}
\usepackage{amssymb}
\usepackage{tikz}
\usetikzlibrary{automata,positioning}
\usepackage{url}
\usepackage{float}
\usepackage{setspace}
%\usepackage{IEEEtran}
\usepackage{filecontents,lipsum}
\usepackage[noadjust]{cite}

\begin{document}
%\raggedright
%\doublespacing

\title{Week 2 Journal}
\author{Rodger Byrd}
\maketitle

Choose a “top” journal or conference in your field with >= 45 papers one or two 2018/2019 issues.   (DO NOT search for papers with keywords.. but can search for journals/conf title).

\section{Process}
The first section in the journal should discuss your process and your learning about the process (do not discuss the content of the papers. 

\section{Raw Notes}
The second section is your raw notes from critical/creative reads as a latex document, and include all the papers in the references. Not expecting/wanting full sentences, grammar, want your raw notes.  If you use handwritten notes, you can take pic and include them as figures. 


The last section is references.. which should start with the citation for the primary conference, plus the citations for have all the papers from the browse, with the timing notes. 


 \cite{1} \cite{2} \cite{4} \cite{5} \cite{6}  \cite{8} \cite{9} \cite{10} \cite{11} \cite{12} \cite{13} \cite{14} \cite{15} \cite{16} \cite{17} \cite{19} \cite{20} \cite{21} \cite{22} \cite{23} \cite{24}

\bibliographystyle{IEEEtran}
\bibliography{references}


%\begin{thebibliography}{9}
%\bibitem{git} 
%Project github, \\\path{https://github.com/rodger79/CS6000}
%\bibitem{jflap} 
%JFLAP, \\\path{http://www.jflap.org/getjflap.html}
%\end{thebibliography}

\end{document}